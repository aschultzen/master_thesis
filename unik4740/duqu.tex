\documentclass[12pt,english,a4paper]{article}

\usepackage[utf8]{inputenc}          % Allows UTF-8 encoded characters in the .tex-file.
\usepackage{babel,csquotes,textcomp} % Set LaTeX to structure the content following international academic standards.

\usepackage{hyperref}
\usepackage{graphicx}
\usepackage{pdfpages}
\usepackage{listings}
\usepackage{wrapfig}
\usepackage{color,colortbl}
\usepackage{lettrine}
\usepackage[font={small,it}]{caption}
\usepackage{multirow}
\usepackage{tabularx}
\usepackage{footnote}

\usepackage[
    backend=biber,
    style=numeric
]{biblatex}
\addbibresource{refs.bib}

\definecolor{mygreen}{rgb}{0,0.6,0}
\definecolor{mygray}{rgb}{0.5,0.5,0.5}
\definecolor{mymauve}{rgb}{0.58,0,0.82}

\definecolor{auxiliryc}{RGB}{70,240,161} %Green

\definecolor{ineffectivec}{RGB}{70,149,240} %purple

\definecolor{effectivec}{RGB}{115,70,240} % Blue

\lstset{ %
  basicstyle=\ttfamily\small,     
  backgroundcolor=\color{white},   % choose the background color
  breaklines=true,                 % automatic line breaking only at whitespace
  captionpos=b,                    % sets the caption-position to bottom
  commentstyle=\color{mygreen},    % comment style
  escapeinside={\%*}{*)},          % if you want to add LaTeX within your code
  keywordstyle=\color{blue},       % keyword style
  stringstyle=\color{mymauve},     % string literal style
}

\title{Report on Duqu: A collection of computer Malware}
\author{Aril Johannes Schultzen}

\begin{document}

\maketitle
\thispagestyle{empty}
\setcounter{page}{0}
%\newpage
\tableofcontents
\thispagestyle{empty}
\setcounter{page}{0}
%\newpage
\thispagestyle{empty}
\setcounter{page}{0}
%\newpage
\clearpage
\setcounter{page}{1}

\begin{abstract}
This report on Duqu (a collection of computer malware) was an assignment given in the course UNIK4740. It is mainly based on \textit{Duqu: A Stuxnet-like malware found in the wild.}\cite{DUQU_BUD} by Boldizsár Bencsáth et al (October 2011) and \textit{W32.Duqu: The precursor to the next Stuxnet}\cite{DUQU_SYMANTEC} by Symantec (November 2011).
\end{abstract}
\newpage

\section{Introduction}
Duqu is a collection of malware discovered by The Laboratory of Cryptography and System Security (CrySyS) of the Budapest University of Technology and Economics in Hungary. They analyzed it and named it Duqu from the prefix "~DQ" that the key logger use to name its files. It is an interesting piece of code despite it being anything but technically astonishing. It is however interesting because of its similarities with \textit{Stuxnet} and \textit{Stuxnets} modular design and how these modules combined can be used to create a targeted thread to control systems in nuclear facilities. It is believed that the creator(s) of Duqu also created \textit{Stuxnet} or at least had access to \textit{Stuxnets} source code.

\section{Prerequisite knowledge}
Though not a technical marvel, Duqu exploits numerous mechanisms and features in the Winodws Operating system. Some of these mechanisms will be explained in this section and should be understood before venturing further into this report.

\subsection{DLL}
A DLL (Dynamic Link Libraries) is Microsoft's implementation of the \textit{shared library} concept used in both Windows and the OS/2 operating system. It can contain both code and data and shares it's file format with the Windows Executable file (EXE). A DLL can be used by multiple programs at the same time. The idea is that it promotes reuse of code while achieving higher memory efficiency. WHY VULNERABLE?

\subsection{Drivers}
A Driver (also known as Device driver) an abstraction layer that provides a software interface to the hardware. A driver can either be written in kernel mode or user mode. When running in kernel mode, the driver has access to every resource and all hardware, this also means that every CPU instruction can be executed and every memory address can accessed. An application written in user-mode can not directly access hardware or memory but has to use APIs instead. This isolation makes a crash in user-mode recoverable instead of catastrophic as in a kernel-mode. WHY VULNERABLE?

\subsection{The Windows registry}
The registry is a database used in Microsoft Windows to store settings and options. It can be considered an alternative to the use of INI files. The use of the registry is not compulsory.

\subsection{RPC}
Remote Procedure Calls (RPC) is a system that allows programmers to write distributed software without worrying about the underlying network code. It is most often used to create a server/client model. \cite{MSRPC}

\section{Installation of Duqu}
The following sections covers one of the most discussed method of delivering and installing Duqu. The process is divided into two, the \textit{Preparation} and the \textit{Installation}.

\subsection{Preparation}
Duqu was delivered to the target by using a specially crafted Microsoft Word document. According to Microsoft the Duqu malware exploits a problem in T2EMBED.DLL which is called by the TrueType font parsing engine \cite{RYAN_ZDNET}. This exploit in the Win32k TrueType font parsing engine allows arbitrary code to run in kernel mode. This arbitrary code will from now on in this report be reference to as the \textit{prep-code}. 

Once the Word document is opened and the exploit triggered, the prep-code will do a check to determine whether or not the target has been infected. This is done by checking the registry for the following value:
\begin{lstlisting}
    HKEY_LOCAL_MACHINE\ SOFTWARE\Microsoft\Windows\CurrentVersion\Internet Settings\Zones\4\"CF1D" 
\end{lstlisting}
Unless the value is found, indicating that the computer is compromised, the prep-code decrypts two files from the Word document:
 \begin{enumerate}
   \item  \texttt{jminet7.sys} or \texttt{cmi4432.sys}, both drivers.
   \item  \texttt{netp191.pnf} or \texttt{cmi4432.pnf}, both DLLs.
  \end{enumerate}
For the sake of simplicity, it is assumed that the \texttt{jminet7.sys} and \texttt{netp191.pnf} is the files used in this attack. According to the CrySys report \cite{DUQU_BUD}, the files are very similar and it is theorized that one of them provides the functionality to utilize the key logger.

The prep-code executes \texttt{jminet7.sys} (driver) which injects the \texttt{netp191.pnf} into \texttt{services.exe}. This behavior is defined by the configuration file \texttt{netp192.pnf} which is loaded and decrypted. Before the prep-code exits, it executes \texttt{netp191.pnf} and overwrite itself with zeros. 

\subsection{Installation}
At this point, the prep-code no longer exists in RAM, and the driver (\texttt{jminet7.sys}) passes the execution to \texttt{netp192.pnf}. This is where things get a little tricky. During the preparation process, \texttt{netp192.pnf} was injected into \texttt{services.exe}. Now, the same code but launched by the driver, decrypts three files from the injected \texttt{netp192.pnf} code. These files are:



\newpage
\printbibliography[title={Complete Bibliography},heading=bibintoc]

\end{document}                     