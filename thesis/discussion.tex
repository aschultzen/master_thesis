\documentclass[../main.tex]{subfiles}

\chapter{Discussion}\label{discussion}

\section{Test results}
The GPS manipulation test as described in chapter \ref{gps_manipulation} demonstrated the ability to quickly identify a GPS disturbance and also demonstrated the ability to protect the atomic clock by disabling the disciplining. The system did, however, fail to steer the atomic clock. When we attempted to reset and debug the system, we encountered a bug in the clock model as well. The bug affected the way the clock model initializes from the configuration file, which meant that the model would need at least 48 hours to rebuild. This is probably just a minor bug but because of the time constraint, we where forced to seize testing. 

\section{Sensor server performance}
During the GPS manipulation tests as described in chapter \ref{gps_manipulation}, the sensor server part of the atomic clock controller worked perfectly. The system was stable and performed as designed over longer periods of time (over 4 days). The system logged data reliably and without fault, accepted connections and was responsive at all times tested. There were no memory leaks or segmentation faults and the system facilitated the detection of our GPS manipulation attempt with low latency. The system would probably react even faster if it received data from the atomic clock and GPS receivers at a higher frequency. The communication with the atomic clock also worked flawlessly, providing data for logging and the model as expected.

\section{Shortcomings in current implementation}
This section is used to discuss some of the shortcomings in the current version of the atomic clock controller and sensor server architecture. Some functions were never implemented and others were not finished in time.

\subsection{Resizing shared memory segments}\label{resizing_shared}
Ideally the shared memory segments containing the client list should be resizeable and its size should depend on the number of connected clients. This proved difficult and I was not able to implement it. M. Kerrisk explains in his book "The Linux Programming Interface" \cite{kerrisk2010linux} that most UNIX implementations do not support resizing of a memory map like the shared memory segments used in the sensor server implementation. There is however a non-portable and Linux-specific system call named \texttt{mremap()} that can be used on Linux systems for this purpose. Unfortunately the address returned by \texttt{mremap()} might be different from the old address to the shared memory segment. This would mean that a pointer inside the shared segment might no longer be valid after a resize operation has been performed. A way to avoid this problem, caused by the remapping would be to use offsets instead of pointers when referring to addresses in the mapped region. While troubleshooting problems I had using \texttt{mremap()}, I stumbled upon a bug report in the the Kernel Bug Tracker \cite{BUG_TRACK} reported by someone with similar issues as I was having. This indicates that the trouble I had, might have been because of a bug in the Linux kernel. I have yet to confirm this but it did convince me to leave the implementation with its shortcomings rather than potentially wasting my time on something way out of my reach. The waste of memory would never by substantial anyway considering that the size of the \texttt{client\_table\_entry} struct is a modest 4664 bytes. 

\subsection{Atomic clock management}
The sensor server should have used a separate process to handle the filters and communication with the atomic clock. This would free the processes handling the connections to clients from dealing with the filters, and make the filter abstraction more complete. This would also make the clock model cleaner. The atomic clock model is already logically separated from the filters associated with it, but because of the way the code is organized the use of the model implies the use of the filter. The atomic clock controller should still communicate with the atomic clock on its own like it does today, retrieving telemetry data, but the aforementioned process could keep track of the atomic clock's discipline status, steering and other functionality, thus creating a more generic way for the system to communicate with the atomic clock. The GPS based filter could greatly benefit from this approach because they today do not do anything but log occurrences where they were triggered. 

\subsection{External MJD calculation}
Modified Julian Day (MJD) is a way to express both date and time as a single number. It's convenient when doing calculations with dates, for example: the difference between the MJD for one day and the next, is exactly 1\footnote{For example, the MJD for 24/10-2016 00:00 is 57685.0 and the MJD for 25/10-2016 00:00 is 57685.0. 12'o clock at the 25/10 would be 57658.5}. Modified Julian Date relies heavily on by the clock model (see \ref{clock_model_harald} for more about the clock model). During testing a python script (see \ref{get_mjd} for more about this script) was used to calculate MJD. This script was scheduled to be replaced by a module written in C, but was left in use because of time constraints. The script is called upon by using \texttt{popen()} which in turn calls \texttt{fork()} to run the script. The clock model is updated every second which means that this script is also invoked every second. This is of course counterintuitive considering that one of our goals for the software was to focus on efficiency.

\subsection{External Atomic clock communication}
For some reason, it proved to be a significant challenge to implement a solution in C to configure, read and write from the atomic clock. The best solution did not even provide a reliable means of communicating with the atomic clock even though communication with the GPS receivers which in theory should have been exactly the same, was no problem. Once time got tight, I made the decision to drop the development of the atomic clock serial communication module written in C, and decided to use \texttt{query\_csac.py} by invoking with \texttt{popen()}. 

\section{Choice of programming language}
The atomic clock controller software was originally planned to be written in Java since this was my most fluent programming language. Java is great language. It is object oriented, it has a garbage collector and a lot of useful libraries. As development commenced it quickly became apparent that some parts of the code would be performance critical and that portability really was not that important anyway. The platform was already chosen and I could not think any reason for it to change. I decided to look at other languages. Because performance was a concern Python was also quickly dismissed as an option. C++ would probably have been the best choice but having never written anything in C before made it sound more exciting and like a nice opportunity to learn something new. During the planning phase of the atomic clock controller development, raspbian-2015-05-07 was the latest build. It came with GCC 4.6.3 which only had experimental support for C11(\cite{GCC11}). With C11 no longer considered an option, C99 was the obvious choice given its attractive features like:
\begin{itemize}
  \item Variable-length arrays.
  \item Single line comments.
  \item snprintf() as standard \cite{C_RATIONAL}.
\end{itemize}

\section{Alternative approaches}\label{da}
When planning on how to execute our proposal, these were among the ideas that came up. 

\subsection{Single computer, many GPS receivers}\label{scmgr}
A single computer is used to run the atomic clock controller software. The atomic clock controller does not include a server/cient model, but the receivers used to collect data are all connected to to the computer through whatever USB ports available or made available by the use of USB hubs. With this approach you are not dependent on a network, but it limits the number of GPS receivers you could connect as the USB specification limits the number possible endpoints to an absolute 127(\cite[pp. 3]{USBTC}) because of addressing. This does not mean that 127 devices can be connected because a single device might use more than one endpoint. It is also worth mentioning that a USB hub might "reserve" multiple endpoints. Depending on the GPS receivers and how they are made, this number might be reduced even further by the power usage of the connected devices. Depending on how far each GPS receiver is distanced from the atomic clock controller, a signal amplifier might be necessary to compensate for the signal attenuation. In some cases where a network is absent, this approach might be only option.

\subsection{Store in database and analyze}
With this approach, the idea of a GPS receiver and Raspberry PI as a single "sensor" unit is the same as with sensor server approach. The difference is that each sensor stores the collected data in a database. The atomic clock controller software monitors the clock directly as with the sensor server approach, but the data in the database is routinely queried and analyzed. The strength with this approach is that data is easily stored, shared and maintained by a single entity. The complexity of the client software would be the same as with the sensor server approach, but the atomic clock controller software could be implemented with less complexity as no client/server architecture or shared memory schemes would be necessary. During planning this approach seemed promising but was rejected because it was thought that it might not be time-sensitive enough. It was also some doubt concerning whether or not the ability to store data to a database actually was important. Once the different filters and algorithms was in place, it turned that the database functionality would have been nice but not of any real importance for the atomic clock controller to perform its tasks.