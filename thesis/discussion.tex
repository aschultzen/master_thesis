\documentclass[../main.tex]{subfiles}

\chapter{Discussion}\label{discussion}
\section{Problems and Challenges}
\subsection{Resizing shared memory segments}\label{resizing_shared}
Ideally, the shared memory segments containing the client list should be resize-able and the size should depend on the number of connected clients. This proved difficult and I was not able to implement it. M. Kerrisk explains in his "The Linux Programming Interface" \cite{kerrisk2010linux}, that most UNIX implementations does not support resizing of a memory map like the shared memory segments used in the Sensor Server. However, there is non-portable and Linux specific system call, \texttt{mremap()} that can be used on Linux system for this purpose. Unfortunately, the address returned by \texttt{mremap()} might be different from the old address to the shared memory segment. This would mean that pointer inside the shared segment might no longer be valid after a resize operation has been done. A way to avoid this problem caused by the remapping, would be to use offsets instead of pointers when referring to addresses in the mapped region. While troubleshooting problems I had using \texttt{mremap()}, I stumbled upon a bug report reported to the Kernel Bug Tracker \cite{BUG_TRACK} describing similar issues as I was having. This indicate that the trouble I was having might be because of a bug in the Linux kernel. I have yet to confirm this, but it did convince me to leave the implementation with its shortcomings rather than potentially wasting my time on something way out of my reach. The waste of memory would never by substantial anyway considering that the size of the \texttt{client\_table\_entry} struct is a mere 4664 bytes. 

\section{What could have been different}
\subsection{Server design}
\begin{itemize}
  \item Poor management of the shared memory space. Searching through the space is taking linear time. Why would you care about performance enough to code in C when you are going to waste that performance on inefficient algorithms?
  \item Redundant readiness checking. The readiness check is a feature that was implemented early in the process when more filters where planned to be implemented. These filters would use cross-checking of different parameters between multiple sensors, and it would therefor be desirable to make sure that all sensor had the latest NMEA data. Unfortunately in the current version of the server, there are no filters actually dependent on this feature, but it has been left this way in case someone else would pick up the development. 
  \item Shared memory is barely necessary in the current version of the Sensor Server. I still believe that later development can benefit from the architecture.
\end{itemize}

\section{Choice of programming language}
The SMACC software was originally planned to be written in Java since this was my most fluent programming language. Java is great language, it's object oriented, it has a garbage collector and a lot of useful libraries. As development started, it quickly became apparent that some parts of the code would be performance critical and that portability really wasn't that important anyway. The platform was already decided and there was no reason to believe that it would change in the near future. As we all know, premature optimization is the root of all evil. Being reluctant to commit a deadly programming sin, i decided to look at other languages. Since performance was a concern, Python was also quickly dismissed as an option. C++ would probably have been the best choice, but having never written anything in C before made it sound more exciting and like a nice opportunity to learn something new. During the planning phase of SMACC development, raspbian-2015-05-07 was the latest build. It came with GCC 4.6.3 which only had experimental support for C11(\cite{GCC11}). With C11 no longer considered an option, C99 was the obvious choice given it's attractive features like:
\begin{itemize}
  \item Variable-length arrays.
  \item Single line comments.
  \item snprintf() as standard (\cite{C_RATIONAL}).
\end{itemize}

\section{Alternative approaches}\label{da}
When planning on how to execute our proposal, these where among the ideas that came up. 

\subsection{Single computer, many GNSS receivers}\label{scmgr}
A single computer is used to run the SMACC software. The SMACC does not include a Server/Cient model, but the receivers used to collect data are all connected to to the computer through whatever USB ports available or made available by the use of USB hubs. With this approach, you are not dependent on a network, but it limits the number of GNSS receivers you could connect as the USB specification limits the number possible endpoints to an absolute 127(\cite[pp. 3]{USBTC}) because of addressing. This does not mean that 127 devices can be connected, a single device might use more than one endpoint. It's also worth mentioning that a USB hub might "reserve" multiple endpoints. Depending on the GNSS receivers and how they are made, this number might be reduced even further by the power usage of the connected devices. Depending on how far each GNSS receiver is distanced from the SMACC, a signal amplifier might be necessary to compensate for the signal attenuation. In some cases where a network is absent, this might be only option.

\subsection{Store in database and analyze}
With this approach, the idea of a GNSS receiver and RASPI as a single "sensor" unit is the same as with Client-server approach. The difference is that it with this approach, each sensor stores the collected data in a database. The SMACC software monitors the clock directly as with the Client-server approach, but the data in the database is routinely queried and analyzed. The strength with this approach is that data is easily stored, shared and maintained by a single entity. The complexity of the client software would be the same as with Client-server approach, but the SMACC software could be implemented with less complexity as no Client-server architecture or shared memory schemes would be necessary. During planning, this approach seemed promising but was rejected because it was thought that it might not be time-sensitive enough. It was also some doubt concerning whether or not the ability to store data to a database actually was important. Once the different filters and algorithms was in place, it turned that the database functionality would have been nice, but not of any real importance for the SMACC to perform it's tasks, and would have been overkill anyway.