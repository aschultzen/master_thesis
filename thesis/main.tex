\documentclass[12pt,english,a4paper]{report}

\usepackage[utf8]{inputenc}          % Allows UTF-8 encoded characters in the .tex-file.
\usepackage{babel,csquotes,textcomp,duomasterforside} % Set LaTeX to structure the content following international academic standards.

\usepackage[toc,page]{appendix}

\usepackage{hyperref}
\usepackage{graphicx}
\usepackage{pdfpages}
\usepackage{listings}
\usepackage{wrapfig}
\usepackage{color,colortbl}
\usepackage{lettrine}
\usepackage[font={small,it}]{caption}
\usepackage{multirow}
\usepackage{tabularx}
\usepackage{footnote}
\usepackage{subfiles}
\usepackage{minted}
\usepackage{fancyvrb}
\usepackage{lscape}
\usepackage{tabularx}
\usepackage{rotating}
\usepackage{verbatim}

\usemintedstyle{bw}

\newmintedfile[pythoncode]{python}{
fontsize=\scriptsize,
fontfamily=tt,
linenos=true,
numberblanklines=true,
numbersep=12pt,
numbersep=5pt,
gobble=0,
frame=leftline,
framerule=0.4pt,
framesep=2mm,
funcnamehighlighting=true,
tabsize=4,
obeytabs=false,
mathescape=false
samepage=false, %with this setting you can force the list to appear on the same page
showspaces=false,
showtabs =false,
texcl=false,
%breaklines = true
}

\newmintedfile[ccode]{c}{
fontsize=\scriptsize,
fontfamily=tt,
linenos=true,
numberblanklines=true,
numbersep=12pt,
numbersep=5pt,
gobble=0,
frame=leftline,
framerule=0.4pt,
framesep=2mm,
funcnamehighlighting=true,
tabsize=4,
obeytabs=false,
mathescape=false
samepage=false, %with this setting you can force the list to appear on the same page
showspaces=false,
showtabs =false,
texcl=false,
%breaklines = true
}

\newmintedfile[makecode]{make}{
fontsize=\scriptsize,
fontfamily=tt,
linenos=true,
numberblanklines=true,
numbersep=12pt,
numbersep=5pt,
gobble=0,
frame=leftline,
framerule=0.4pt,
framesep=2mm,
funcnamehighlighting=true,
tabsize=4,
obeytabs=false,
mathescape=false
samepage=false, %with this setting you can force the list to appear on the same page
showspaces=false,
showtabs =false,
texcl=false,
%breaklines =true
}

% Making the whole paragraph biz possible
%\usepackage{titlesec}
%\setcounter{secnumdepth}{4}
%\titleformat{\paragraph}
%{\normalfont\normalsize\bfseries}{\theparagraph}{1em}{}
%\titlespacing*{\paragraph}
%{0pt}{3.25ex plus 1ex minus .2ex}{1.5ex plus .2ex}

% Adding the bib
\usepackage[
    backend=biber,
    style=numeric,
    sorting=none
]{biblatex}
\addbibresource{refs.bib}

\definecolor{mygreen}{rgb}{0,0.6,0}
\definecolor{mygray}{rgb}{0.5,0.5,0.5}
\definecolor{mymauve}{rgb}{0.58,0,0.82}
\definecolor{auxiliryc}{RGB}{70,240,161} %Green
\definecolor{ineffectivec}{RGB}{70,149,240} %purple
\definecolor{effectivec}{RGB}{115,70,240} % Blue

\graphicspath{{./graphics/}}

% Setting tty font?
%\lstset{ %
%  basicstyle=\ttfamily\small,     sa 45 csac user guide
%  backgroundcolor=\color{white},   % choose the background color
%  breaklines=true,                 % automatic line breaking only at whitespace
%  captionpos=b,                    % sets the caption-position to bottom
%  commentstyle=\color{mygreen},    % comment style
%  escapeinside={\%*}{*)},          % if you want to add LaTeX within your code
%  keywordstyle=\color{blue},       % keyword style
%  stringstyle=\color{mymauve},     % string literal style
%  showspaces=false,
%  showstringspaces=false,
%}

\lstset{language=C,
  basicstyle=\ttfamily\scriptsize,
  keywordstyle=\color{blue}\ttfamily,
  stringstyle=\color{red}\ttfamily,
  commentstyle=\color{green}\ttfamily,
  breaklines=true,
  showspaces=false,
  showstringspaces=false,
  numbers=left,                    % where to put the line-numbers; possible values are (none, left, right)
  numbersep=5pt,    
  frame=single,                 % how far the line-numbers are from the code
  numberstyle=\tiny\color{mygray}, % the style that is used for the line-numbers
  rulecolor=\color{black}         % if not set, the frame-color may be changed on line-breaks 
}
\setcounter{secnumdepth}{5}
\newenvironment{code}{\captionsetup{type=listing}}{}

\title{Spoof proof GPS timing}
\subtitle{A detection and mitigation system for GPS time spoofing}
\author{Aril Johannes Schultzen}
\begin{document}
\duoforside[dept={Spoof proof GPS timing},
program={Informatikk: programmering og nettverk},
long]

\begin{abstract}
The goal of this project is to create a prototype of a smart atomic clock controller running on commodity hardware, capable of spoof proofing a GPS controlled atomic clock, i.e harden it against GPS jamming and spoofing attacks. The project aims to use a multi-layered approach to evaluate the integrity of GPS signals and to use knowledge about the atomic clock and its behavior when GPS disciplined, to detect and mitigate a spoofing attack. We aim to demonstrate the efficiency of network enabling commodity GPS receivers and connecting them to a atomic clock controller by using a client/server architecture facilitating GPS and atomic clock data analysis. The preliminary GPS manipulation tests demonstrates the feasibility of the system.
\end{abstract}

% Where = hvor
% were = var

% Hackish code to keep counters and white pages at bay
\thispagestyle{empty}
\setcounter{page}{0}
\tableofcontents
\thispagestyle{empty}
\setcounter{page}{0}
\thispagestyle{empty}
\setcounter{page}{0}
\clearpage
\setcounter{page}{1}

% Here goes all the chapters
% ==================================================

% Ikke gjort:
% - Alle figurer må være på samme side som de prates om. De må også referes til i teksten.
% - Bruk referenselista. Ikke referer i teksten.s
% - All the figures have CSAC written in them
% - Jekk ned bruken av refs. Blir litt mye noen ganger.

\subfile{intro}
\subfile{proposal}
\subfile{hw}
\subfile{software}
\subfile{interfacing}
\subfile{spoof_test}
\subfile{discussion}
\subfile{conclusion}
\subfile{appendix}

%===================================================

\newpage
\printbibliography[title={Complete Bibliography},heading=bibintoc]

\end{document}                    