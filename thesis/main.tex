\documentclass[12pt,english,a4paper]{report}

\usepackage[utf8]{inputenc}          % Allows UTF-8 encoded characters in the .tex-file.
\usepackage{babel,csquotes,textcomp,duomasterforside} % Set LaTeX to structure the content following international academic standards.

\usepackage[toc,page]{appendix}

\usepackage{hyperref}
\usepackage{graphicx}
\usepackage{pdfpages}
\usepackage{listings}
\usepackage{wrapfig}
\usepackage{color,colortbl}
\usepackage{lettrine}
\usepackage[font={small,it}]{caption}
\usepackage{multirow}
\usepackage{tabularx}
\usepackage{footnote}

\usepackage{subfiles}

% Making the whole paragraph biz possible
%\usepackage{titlesec}
%\setcounter{secnumdepth}{4}
%\titleformat{\paragraph}
%{\normalfont\normalsize\bfseries}{\theparagraph}{1em}{}
%\titlespacing*{\paragraph}
%{0pt}{3.25ex plus 1ex minus .2ex}{1.5ex plus .2ex}

% Adding the bib
\usepackage[
    backend=biber,
    style=numeric
]{biblatex}
\addbibresource{refs.bib}

\definecolor{mygreen}{rgb}{0,0.6,0}
\definecolor{mygray}{rgb}{0.5,0.5,0.5}
\definecolor{mymauve}{rgb}{0.58,0,0.82}
\definecolor{auxiliryc}{RGB}{70,240,161} %Green
\definecolor{ineffectivec}{RGB}{70,149,240} %purple
\definecolor{effectivec}{RGB}{115,70,240} % Blue

\graphicspath{{./graphics/}}

% Setting tty font?
\lstset{ %
  basicstyle=\ttfamily\tiny,     
  backgroundcolor=\color{white},   % choose the background color
  breaklines=true,                 % automatic line breaking only at whitespace
  captionpos=b,                    % sets the caption-position to bottom
  commentstyle=\color{mygreen},    % comment style
  escapeinside={\%*}{*)},          % if you want to add LaTeX within your code
  keywordstyle=\color{blue},       % keyword style
  stringstyle=\color{mymauve},     % string literal style
  showspaces=false,
  showstringspaces=false,
}

\title{GPS Spoofing}
\subtitle{A detection and mitigation system}
\author{Aril Schultzen}
\begin{document}
\duoforside[dept={Institutt for informatikk},
program={Informatikk: programmering og nettverk},
long]

\begin{abstract}
Abstract goes here.
\end{abstract}

\chapter*{Foreword}
Here goes foreword

% Hackish code to keep counters and white pages at bay
\thispagestyle{empty}
\setcounter{page}{0}
%\newpage
\tableofcontents
\thispagestyle{empty}
\setcounter{page}{0}
%\newpage
\thispagestyle{empty}
\setcounter{page}{0}
%\newpage
\clearpage
\setcounter{page}{1}

% Here goes all the chapters
% ==================================================

\subfile{intro}

% Chapter about the csac and it's connectivity?

\newpage
\chapter{Method}
\section{Data acquisition}
In order to create an accurate clock-model of the CSAC, it was necessary to log data from it while it was running in a disciplined mode. In the disciplined mode, the CSAC will correct it's frequency based on either a 1 PPS (Pulse per second) signal or 10 MHz signal. The Ublox-M8T GPS receiver was specifically designed for timing purposes and was used in this instance to provide the 1 PPS signal for disciplining purposes. A similar approach was used in order to collect GPS data. Data from two Ublox-M8T was gathered over the same time period as the data gathered from the CSAC. By gathering the data over the same period, it was possible to detect any correlation between the time solved by the GPS receivers and any frequency adjustments done by the CSAC. It also provided valuable data that could be used to tune the spoofing detection algorithms in the CSAC SMACC. The data gathering was done by simple Python scripts (\ref{CL} and \ref{GL}) running on a computer connected to the receivers and the CSAC (\ref{CLS})

\newpage
\section{Programming}

\newpage
\chapter{Testing}

\newpage
\chapter{Results and discussion}
\section{Different approaches}

\newpage
\chapter{Conclusion}

\newpage
\subfile{appendix}

%===================================================

\newpage
\printbibliography[title={Complete Bibliography},heading=bibintoc]

\end{document}                    